%\documentclass[secnumarabic,twocolumn,preprintnumbers,amsmath,amssymb,aps]{revtex4}
\documentclass{rmaa}

\usepackage{paralist}

% These are used in one of the graphics examples
\usepackage{psfrag,color}

% Allow accented characters to be entered directly
\usepackage[latin1]{inputenc}


\usepackage{natbib}
%\usepackage{bm}
%\usepackage{latexsym}
%\usepackage{eufrak}
\usepackage{epsfig}
%\usepackage{latexsym}
\usepackage{amsmath}
%\usepackage{epsfig}
%\usepackage{ifpdf}
%\usepackage{graphicx}
%\usepackage{dcolumn}


\hyphenation{pa-ra-me-ter}
\hyphenation{In-fla-tio-na-ry}
\hyphenation{a-cce-ssi-ble}
\hyphenation{cos-mo-lo-gi-cal}
\hyphenation{a-pproach}
\hyphenation{do-mi-na-ted}
\hyphenation{ex-pli-ci-tly}
\hyphenation{Hu-bble}
\hyphenation{re-pre-sen-ted}
\hyphenation{lo-ga-rith-mic}
\hyphenation{Ho-we-ver}
\hyphenation{di-ffe-rent}


\def\beq{\begin{equation}}
\def\eeq{\end{equation}}
\def\bea{\begin{eqnarray}}
\def\eea{\end{eqnarray}}
\def\cal{\mathcal}




%\begin{document}
  \title{Constraining Cosmological Inflation}


\author{
J. Alberto V\'azquez \altaffilmark{1}, Luis, E. Padilla \altaffilmark{2,3}, Tonatiuh Matos \altaffilmark{2,3}}

\altaffiltext{1}{Kavli Institute for Cosmology / Cavendish Laboratory, Cambridge, UK.}

\altaffiltext{2}{Departamento de F\'{\i}sica, Centro de
Investigaci\'on y de Estudios Avanzados del IPN, M\'exico.}
  
  \altaffiltext{3}{Instituto Avanzado de
Cosmolog\'ia (IAC), http://www.iac.edu.mx/}


\shortauthor{Vazquez, Padilla \& Matos}

\shorttitle{Constraining Cosmological Inflation}


\fulladdresses{
\item J. Alberto V\'azquez: Kavli Institute for Cosmology, Madingley Road, Cambridge CB3 0HA, UK.
\\Astrophysics Group, Cavendish Laboratory, JJ Thomson Avenue, Cambridge CB3 0HE, UK. 
(jv292@cam.ac.uk)
\item Luis E. Padilla: Departamento de F\'{\i}sica, Centro de
Investigaci\'on y de Estudios Avanzados del IPN, AP 14-740,07000 M\'exico D.F., M\'exico.
(epadilla@fis.cinvestav.mx)
\item Tonatiuh Matos: Departamento de F\'{\i}sica, Centro de
Investigaci\'on y de Estudios Avanzados del IPN, AP 14-740,07000 M\'exico D.F., M\'exico.
(tmatos@fis.cinvestav.mx)
}


\listofauthors{J. Alberto V\'azquez,  \& Tonatiuh Matos}

\indexauthor{V\'azquez J.A.}
%\indexauthor{Matos T.}




\abstract{
The aim of this paper is to provide a qualitative introduction to the inflationary theory
and its relation with current cosmological observations.
Inflation solves some of the fundamental problems which challenge the 
Standard Big Bang cosmology i.e. Flatness, Horizon and Monopole problem,
and additionally explains the initial conditions for the 
Large-Scale Structure observed in the Universe, such as galaxies.
We describe the general properties of this solution carried out by
a single scalar field.
Finally, with the use of current and future surveys,
we show constraints on the inflationary parameters $(n_{\rm s},r)$ which allow us to 
make the connection between the theoretical and observational cosmology. 
In this way, with the latest observational results, it is possible to choose or 
at least to constrain the right inflationary model, parameterised by the 
scalar field potential $V(\phi)$.
}


\resumen{
El objetivo de este art\'iculo es ofrecer una introducci\'on cualitativa 
a la teor\'ia de la inflaci\'on cosm\'ologica y su relaci\'on con las observaciones
cosmol\'ogicas actuales. Inflaci\'on resuelve algunos de los problemas fundamentales que 
desaf\'ian al modelo est\'andar de la cosmolog\'ia (Big Bang), 
por ejemplo, el problema de la Planicidad, Horizonte y la inexistencia de Monopolos,
y adem\'as de resolver estos problemas, explica el origen de la estructura a gran escala del 
Universo, como son las galaxias.
Se describen las caracter\'isticas generales de esta soluci\'on llevada a cabo por
un campo escalar.
Por \'ultimo, con el uso de recientes (y futuros) estudios, se presentan constricciones de los par\'ametros 
inflacionarios  $(n_{\rm s},r)$ que nos permitir\'an realizar la conexi\'on 
entre la teor\'ia y las observaciones cosmol\'ogicas. De \'esta manera, con los \'ultimos 
resultados observacionales, es posible elegir o al menos limitar el modelo inflacionario correcto, 
parametrizado por el potencial de campo escalar $ V(\phi) $.
}

\addkeyword{cosmology: cosmological parameters}
\addkeyword{cosmology: observations}
\addkeyword{cosmology: inflation}

\begin{document}
% Typeset article header

  \maketitle

%%%%%%%%%%%%%%%%%%%%%%%%%%%%%%%%%%%%%%%% %%%%%%%
\section{Introduction}
%%%%%%%%%%%%%%%%%%%%%%%%%%%%%%%%%%%%%%%% %%%%%%%

Nowadays, the Standard Big Bang (SBB) cosmology is the most accepted model describing 
the central features of the observed Universe. This model has been successfully proved on 
cosmological levels, for instance, numerical simulations on the structure formation of galaxies,
galaxy cluster \textcolor{red}{and the abundance of primordial elements are some predictions that} are in good agreement with astronomical observations \citep{Teg, Sping,Kolbbo}. 
The SBB model also predicts the fluctuations on the temperature observed in the Cosmic 
Microwave Background radiation (CMB) with high degree of accuracy: 
inhomogeneities of about one part in one hundred thousand \citep{Komat}.
 These two predictions, amongst many others, are the great success of 
the SBB cosmology. Nevertheless, %it is still a model and cannot be considered 
%as the final theory. That is, 
when we look at cosmological observations, there might seem to exist certain 
inconsistencies or unexplained features in contrast   with expected by 
the theory. Some of these unsatisfactory aspects led to the 
emergence of the inflationary model \citep{Guth, Linde, Linde2, Steinhardt}.

In this work, we briefly present some of the relevant shortcomings the standard 
cosmology is dealing with and a short review is carried out about the scalar fields as  
promising solution. Moreover, it is shown how an inflationary single-field model can be completely described 
by only specifying its potential form $V(\phi)$. 
Based on the slow-roll approximation, it is found that the observational parameters 
which allow us to make the connexion with experiments are given by: 
the amplitude of density perturbation $\delta_H$, the scalar spectral index $n_{\rm s}$
and the tensor-to-scalar ratio $r$.
Finally, the theoretical predictions for different scalar field potentials are shown and 
compared with current observational data on the phase-space parameter $n_{\rm s}-r$, 
thus, constraining the number of candidates and making predictions on the shape of $V(\phi)$. 


%%%%%%%%%%%%%%%%%%%%%%%%%%%%%%%%%%%%%%%% %%%%%%%
\section{Problems with the old
cosmological model}
%%%%%%%%%%%%%%%%%%%%%%%%%%%%%%%%%%%%%%%% %%%%%%%

Before starting with the theoretical description, let us consider some assumptions on which the
SBB model is built \citep{Coles}:

1) The physical laws at the present time can be extrapolated further back in time and also be
 considered as valid in the early Universe. In this context, gravity is described by
 the theory of General Relativity without a cosmological constant ($\Lambda$) up to the Plank era.  

 2) The cosmological principle holds: ``There do not exist preferred places in the Universe".
 This is telling us that the properties of the Universe at large-scale must be the homogeneity and isotropy,
both of them encoded on the Friedmann-Robertson-Walker (FRW) metric

\begin{equation}
 ds^2= -dt^2 + a^2(t)\left[ \frac{dr^2}{1-kr^2} +r^2 (d\theta^2 +sin^2\theta\, d\phi^2) \right],
\end{equation}

\noindent
where $(t,r,\theta,\phi)$ describe the time-polar coordinates; the spatial curvature is given by the 
constant $k$ and the scale-factor $a(t)$ represents the physical size of the Universe.
  
 3) The anisotropic Universe is well described by a linear expansion of the metric about the 
 FRW background:
 
\beq \label{eq:metric}
g_{\mu \nu}(\textbf{x},t)= g_{\mu \nu}^{FRW}(\textbf{x},t)+h_{\mu \nu}(\textbf{x},t).
\eeq

To avoid long calculations and make this article accessible to young scientists, many 
technical details have been omitted or simplified; 
we encourage the reader to check out the vast amount of literature about the inflationary theory
 \citep{LiddleLyth, Liddle, Kolbbo, Dodelson, Lindeb}.
\\

To describe the general properties of the Universe, we assume its dynamics is governed by a source treated as a
perfect fluid with pressure $p(t)$ and density $\rho(t)$. Both quantities may often be related
via an equation of state $p=p(\rho)$. Some of the well studied cases are  

\begin{eqnarray}
p&=& \frac{\rho}{3} \qquad \qquad {\rm radiation}, \nonumber \\
p&=&0 \qquad \qquad \quad {\rm dust}, \\
p&=&-\rho \qquad \qquad \quad \Lambda. \nonumber
\end{eqnarray}

\noindent
The Einstein equations for these kind of constituents, neglecting the cosmological 
constant $\Lambda$ contribution, are given by:

\noindent
The {\bf Friedmann equation}
\beq\label{eq:Friedmann}
H^2  \equiv   \frac{8\pi}{3 m^2_{Pl}} \rho  - \frac{k}{a^2}, 
\eeq

\noindent
and the {\bf acceleration equation}
\beq \label{eq:Acce}
\frac{\ddot{a}}{a}  =   - \frac{4\pi }{3 m^2_{Pl}} (\rho +3p).
\eeq

\noindent
The energy conservation for the fluids is described by the {\bf fluid equation}

\begin{equation}
\dot \rho + 3H(\rho + p)=0,
\end{equation}

\noindent
where overdots mean time derivative and $H \equiv \dot a/a$ defines the \textit{Hubble parameter}. Hereafter we employ natural units
$c=\hbar=1$; the Planck mass $m_{Pl}$ is related with the gravitational constant $G$ through $G\equiv m^{-2}_{Pl}$.


We notice, from (\ref{eq:Friedmann}), that for a particular 
Hubble parameter there exists a particular density for which the universe is spatially flat 
$(k=0)$. This is known as the {\it critical density} $\rho_c$ and is given by

\beq
\rho_c(t)\, =\, \frac{3 m^2_{Pl} \,H^2}{8\pi},
\eeq

\noindent
where $\rho_c$ is a function of time due to the presence of $H$. In particular, its present 
value is denoted as $\rho_{c,0}=1.88\, h^2\, \times 10^{-26}$ kg m$^{-3}$, or in terms of more 
convenient units taking into account large scales in the 
Universe,   $\rho_{c,0}= 2.78 \, h^{-1}\, \times 10^{11} M_{\odot} /(h^{-1} {\rm Mpc})^3 $;
with the solar mass denoted by $M_{\odot}=1.988\times 10^{33}$g
 and $h$ parameterises the present value of the Hubble parameter as

\beq
H_0 = 100 h\, {\rm km\,s}^{-1}{\rm Mpc}^{-1} = \frac{h}{3000}{\rm Mpc}^{-1}.
\eeq

\noindent
The latest value for the Hubble parameter obtained by the \textit{Hubble Space Telescope}
is quoted to be \citep{Riess}

\beq
 H_0= 74.2 \pm 3.6 \, {\rm km s}^{-1} {\rm Mpc}^{-1}.
\eeq
%
Usually, it is more useful to measure the energy density as a fraction of the critical density,
defining the \textit{density parameter} $\Omega_i= \rho_i / \rho_c$. The label $i$ represents
different constituents of the Universe, such as radiation or matter. 
Then, the Friedmann equation (\ref{eq:Friedmann}) can then be written in such a way to
directly relate the density parameter and the curvature of the Universe as

\begin{equation} \label{eq:curvature}
 \Omega -1={k \over a^2H^2}.
\end{equation}
%
Thus, the correspondence between the matter content $\Omega$ and the space-time 
curvature for different $k$ values is: 
\begin{itemize}
\item Open Universe : $0<\Omega<1: \, k<0: \, \rho<\rho_c$. 
\item Flat Universe       : $\Omega=1: \, k=0: \, \rho=\rho_c$. 
\item Closed Universe: $\Omega>1: \, k>0: \, \rho>\rho_c$.
\end{itemize}

   
\noindent
Current cosmological observations, based on the standard model, suggest the present value of 
$\Omega$ is \citep{Komat}

\beq \label{eq:Omega}
\Omega_0=1.00\pm 0.002,
\eeq
%
that is, the present Universe is very nearly flat.
\\

%\vspace{1cm}
%%%%%%%%%%%%%%%%%%%%%%%%%%%%%%%%%%%%%%%%

\begin{center}
\textbf{\large Shortcomings}
\end{center}

\vskip 10pt

\textbf{Flatness problem}
\vskip 10pt
%%%%%%%%%%%%%%%%%%%%%%%%%%%%%%%%%%%%%%%%


We notice that an special case of equation (\ref{eq:curvature}) is $\Omega=1$. 
If at the beginning the Universe was 
perfectly flat,  then it remains so for all time. Nevertheless, a flat geometry is an unstable
critical situation, that is, even a tiny deviation from it, $\Omega$ would evolve 
quite different and very quickly the Universe would become more curved. 
This can be seen as a consequence due to $aH$ is a decreasing function of time 
during radiation or matter domination epoch.
We observe it from (\ref{eq:curvature}):
%

\bea
&\mid \Omega-1\mid & \, \propto \, t  \hspace{1cm} {\rm radiation\,\, domination},\nonumber \\ 
&\mid \Omega-1\mid & \, \propto \, t^{2/3}  \hspace{1cm} {\rm dust\,\, domination}. \nonumber
\eea

\noindent
Since the present age of the Universe is estimated to be $t_0 \simeq 10^{17} 
\, $sec \citep{Larson}, from the above equation we can 
deduce the required value of $\mid \Omega-1\mid$ at different early-times in order to 
obtain the correct value of spatial-geometry at present time. For instance, let us consider some 
particular epochs in a nearly flat universe,

\begin{itemize}
\item Decoupling  $(t \simeq 10^{13}\, {\rm sec})$, we need $\mid \Omega-1 \mid$ $\le 10^{-3}$.
\item  Nucleosynthesis $(t \simeq 1\, {\rm sec})$, we need $\mid \Omega-1 \mid$ $\le 10^{-16}$.
\item  Planck epoch $(t \simeq 10^{-43}\, {\rm sec})$, we need $\mid \Omega-1 \mid$  $\le 10^{-64}$.
\end{itemize}
% 
%
Because there is no reason to prefer a Universe with critical density, hence
$\mid \Omega-1\mid$ should not necessarily be exactly zero. 
Consequently, at early times 
$\mid \Omega-1\mid$ have to be fine-tuned extremely close to zero in order to reach 
its actual observed value.



%%%%%%%%%%%%%%%%%%%%%%%%%%% %%%%%%%%%%%%%%%%%%%%
\vskip 16pt
\textbf{Horizon problem} 
\vskip 10pt
%%%%%%%%%%%%%%%%%%%%%%%%%%%%%%%%%%%%%%%%%%%%%%%%

The horizon problem is one of the most important problems in the Big Bang model,
it refers to the communication between different regions of the Universe. 
%
Bearing in mind the \textit{Anthropic Cosmological Principle} holds 
\citep{Barrow, Coles}, which is intimately connected
with the existence of the Big Bang, the age of the Universe is a finite quantity and hence
even light should have only travelled a finite distance by any given time. 


According the standard cosmology, photons decoupled from the rest of the 
components at temperatures about $T_{dec}\approx 0.3\, eV$ at redshift
$z_{dec} \approx 1100$, from this time on photons free-streamed and travelled basically
 uninterrupted until reach us, giving rise to the region known as the Observable Universe.
 This spherical surface at which decoupling process occurred is called 
\textit{surface of last scattering}.
The primordial photons are responsible for the CMB observed today, then looking at the
fluctuations is analogous of taking a picture of the universe at this time 
($t_{dec}\approx 380,000$ yrs old), see Figure \ref{fig:wmap5}.

\begin{figure}[ht] 
\centerline{ \epsfxsize=210pt \epsfbox{WMAP7.png} }
\caption{Temperature fluctuations observed in the CMB using 
 WMAP  seven year data \citep{Gold}. }
\label{fig:wmap5}
\end{figure}

Figure \ref{fig:wmap5} shows light seen in all directions of the sky, 
these photons randomly distributed have nearly the same temperature $T_0= 2.725$ K 
plus small fluctuations (about one part in one hundred thousand).
 As we have already noted, being at the same 
temperature is a property of thermal equilibrium, thus observations are easily explained 
if different regions of the sky have been able to interact and moved towards thermal 
equilibrium. In other words, the isotropy observed in the CMB might imply that the radiation was 
homogeneous and isotropic in regions located on the last scattering surface.

Oddly, the comoving horizon over which causal interactions occurred before  
photons decoupled was significantly smaller than the comoving distance 
that radiation travelled after decoupling.
This means that photons coming from separated sky regions by more than the
horizon scale at last scattering, typically about $2^\circ$, would not 
have been  able to interact and established thermal equilibrium before decoupling. 
A simple calculation displays that at decoupling time the comoving horizon was
90 $h^{-1}$ Mpc and would be stretched up to 2998 $h^{-1}$ Mpc at present time.
Then, the microwave background should have consisted of about $10^4$ causally 
disconnected regions.  
Therefore, the Big Bang model by itself does not offer an explanation on why
temperatures seen in opposite sides of the sky are so accurately the same; the homogeneity
must have been part of the initial conditions. 
\\

On the other hand, the microwave background is not perfectly isotropic, but instead exhibits
small fluctuations as detected by, initially, the Cosmic Background Explorer satellite (COBE) \citep{Smooth} and 
now, with improved measurements by the Wilkinson Microwave Anisotropy Probe (WMAP)
\citep{wmap5, Larson}. These tiny irregularities are thought to be the `seeds' that grew 
up until become the structure nowadays observed in the Universe. 
\\


%%%%%%%%%%%%%%%%%%%%%%%%%%%%%%%%%%%%%%%%%%%%%%%%
%%%%%%%%%%%%%%%%%%%%%%%%%%% %%%%%%%%%%%%%%%%%%%%
\vskip 16pt
\textbf{Monopole problem} 
\vskip 10pt
%%%%%%%%%%%%%%%%%%%%%%%%%%%%%%%%%%%%%%%%%%%%%%%%

Following the line to find out the simplest theory to describe entirely the laws of the Universe,
several models in particle physics were suggested to unified three of the 
four forces presented in the Standard Model of Particle Physics (SM): strong force, described
by the group $SU(3)$, weak and electro-magnetic force, with associated group $SU(2)\otimes U(1)$. 
These classes of theories are called \textit{Grand Unified Theories (GUT)} \citep{Georgi}.
 %
 An important point to mention in favour of GUT,  is that they are the only theories which 
 predict the equality electron-proton charge magnitude. Also, there are good reasons to 
 believe that the origin of \textit{baryon asymmetry} might have been generated by GUTs \citep{Kolb83}.
\\

Basically, these kind of theories assert that in the early Universe ($t \sim 10^{-43}\, $sec), 
at highly extreme temperatures ($T_{GUT}\sim 10^{32} \, $K), existed a unified or 
\textit{symmetric phase} described by a group $G$. As the Universe
temperature dropped off, it went through many different phase transitions until reach 
the matter particles such as electrons, protons, neutrons, photons.
%
When a phase transition happens, its symmetry is broken, thus the symmetry group changes by itself.
For instance: 
 \begin{itemize}
 \item GUT transition: $$G \to SU(3)\,\otimes\, SU(2)\, \otimes \, U(1).$$
 \item Electroweak transition: $$SU(3)\,\otimes\, SU(2)\, \otimes \, U(1) \to SU(3)\, \otimes \, U(1).$$ 
\end{itemize}

\noindent
The phase transitions have plenty of implications, one of the most important is the
\textit{topological defects} production, that depends on the type of symmetry breaking 
and the spatial dimension \citep{Vilenkin}, some of them are:   

\begin{itemize}
\item Monopoles (zero dimensional).
\item Strings (one dimensional).
\item Domain Walls (two dimensional).
\item Textures (three dimensional).
\end{itemize}

\noindent
Therefore, monopoles are expected to emerge as a consequence of unification models. 
Besides that, from particle physics models, monopoles would have a mass of $10^6$ orders the proton 
mass. Hence, based on their non-relativitic character, 
a crude calculation predicts an extremely high abundance at present time \citep{Coles}
$$
\Omega_M \simeq 10^{16}.
$$
%
According to this prediction, the Universe would be dominated by magnetic monopoles.
However, in contrast with current observations, no one
has found anyone \citep{Ambrosio02}. 
\\


%%%%%%%%%%%%%%%%%%%%%%%%%%%%%%%%%%%%%%%%%%%%%%%%
\section{Cosmological Inflation}
\vskip 6pt
%%%%%%%%%%%%%%%%%%%%%%%%%%%%%%%%%%%%%%%%%%%%%%%%

The inflationary model offers the most elegant way so far proposed to solve the problems
aforementioned and therefore to understand why the universe is so remarkably in agreement 
with the standard cosmology. It does not replace the Big Bang model, but rather it is considered 
as an `auxiliary patch' which occurred at the earliest stages without disturbing any of its successes.
\\

\textit{Inflation} is defined as the epoch in the evolution of the Universe in which the scale factor 
is quickly accelerated in just a fraction of a second:

\bea \label{eq:inflation}
{\rm INFLATION} &\Longleftrightarrow&~~\ddot a>0 \\
&\Longleftrightarrow& \frac{d}{dt}\left( \frac{{1}}{aH}\right)<0. \label{eq:inflation2}
\eea

\noindent
The last term corresponds to the comoving Hubble length 
$1/(aH)$ which is interpreted as the observable 
Universe becoming smaller during inflation. This process allows our observable
region to lie within a region that was inside the Hubble radius at the beginning of inflation, 
in \citet{Liddle2} words ``is something
similar to zooming in on a small region of the initial universe", see
Figure \ref{fig:Liddle}.
\\

\begin{figure}[ht] 
\centerline{ \epsfxsize=180pt \epsfbox{infla_1} }
\caption{Schematic behaviour 
of the comoving Hubble radius during the inflationary period}% \citep{LiddleLyth}.}
\label{fig:Liddle}
\end{figure}

From the acceleration equation (\ref{eq:Acce}) we can write the condition for inflation in 
terms of the material required to drive the expansion

\beq
\ddot a>0 \Longleftrightarrow (\rho +3p)<0.
\label{gg}
\eeq

\noindent
Because in standard physics it is always postulated $\rho$ as positive, 
to satisfy the acceleration condition it is necessary for the overall pressure to have 

\beq 
{\rm INFLATION} ~~ \Longleftrightarrow~~p<-\rho/3.
\eeq

\noindent
Nonetheless, neither a radiation nor a matter dominated epoch satisfies such condition. 
Let us postpone for a while the problem of finding a `candidate' which may satisfy this inflationary condition.



%%%%%%%%%%%%%%%%%%%%%%%%%%%%%%%%%%%%%%%%%%%%%%%%
\subsection{Solution of the Big Bang Problems}
%%%%%%%%%%%%%%%%%%%%%%%%%%%%%%%%%%%%%%%%%%%%%%%%

\noindent
\textbf {Flatness problem}
\vskip 6pt
%%%%%%%%%%%%%%%%%%%%%%%%%%%%%%%%%%%%%%%%%%%%%%%%

If this brief period of accelerated expansion occurred, then it is possible that the 
aforementioned problems of the Big Bang could be solved. 
A typical solution is a universe possessing a cosmological constant $\Lambda$, which can be 
interpreted as a perfect fluid with equation of state $p=-\rho$. Having this condition, 
since H is constant, we observe from the Friedmann equation (\ref{eq:Friedmann}) that 
the universe is exponentially expanded:
%
\beq
a(t)\propto \exp(Ht),
\eeq
then, the condition  
(\ref{eq:inflation2}) is naturally fulfilled. This epoch is called \textit{de Sitter stage}.
However, postulating a cosmological constant might create more problems than
solve by itself \citep{Carrol01}.
\\

Let us look what happens when a general solution is considered.
If somehow there was an accelerated expansion, $1/(aH)$ tends to be smaller on time and hence,
by the expression (\ref{eq:curvature}), $\Omega$ is driven towards the unity rather than away from it. 
Then, we may ask ourselves by how much should $1/(aH)$ decrease. 
If the inflationary period started at time $t=t_i$ 
and ended up approximately at the beginning of the radiation dominated era ($t=t_f$), then 

$$
\mid \Omega -1\mid_{t=t_f}\sim10^{-60},
$$
and
\beq
\frac{\mid \Omega -1\mid_{t=t_f}}{\mid \Omega -1\mid_{t=t_i}}= \left( a_i \over a_f \right)^2\equiv e^{-2N}.
\eeq
\\
%
So, the required condition to reproduce the value of $\Omega_0$ today is 
that inflation lasted for at least $N\equiv\ln a \gtrsim 60 $, then $\Omega$ will be
extraordinarily close to one that we still observe it today.
In this sense, inflation magnifies the curvature radius of the universe, so 
locally the universe seems to be flat with a great precision.


%%%%%%%%%%%%%%%%%%%%%%%%%%%%%%%%%
\vskip 16pt
\noindent
\textbf{Horizon problem}
\vskip 6pt
%%%%%%%%%%%%%%%%%%%%%%%%%%%%%%%%%%%%%%%%%%%%%%%%


As we have already seen, during inflation the universe expands drastically and there is a 
reduction in the comoving Hubble length. This allowed a tiny region located inside the 
Hubble radius to evolve and constitute our present observable Universe. 
Fluctuations were hence stretched outside of the horizon during inflation and re-entered the horizon 
in the late Universe, see Figure \ref{fig:Liddle}. Scales that were outside the horizon at CMB decoupling 
were in fact inside the horizon before inflation. The region of space corresponding to the 
observable universe therefore was in thermal equilibrium before inflation and the uniformity 
of the CMB is essentially explained.
\\


%%%%%%%%%%%%%%%%%%%%%%%%%%%%%%%%%%%%%%%
\vskip 16pt
\noindent
\textbf{Monopole problem}
\vskip 6pt

The monopole problem was initially the motivation to develop the inflationary
cosmology \citep{Guth2}.
%
During the inflationary epoch, the Universe led to a dramatic expansion
over which the density of the unwanted particles were diluted away. Generating enough
expansion, the dilution made sure the particles stayed completely out of our observable Universe
making pretty difficult to localise a single monopole.     

%%%%%%%%%%%%%%%%%%%%%%%%%%%%%%%%%%%%%%%%%




%%%%%%%%%%%%%%%%%%%%%%%%%%%%%%%%%%%%%%%%%%%%%%%%
\section{Single-field inflation}
\vskip 6pt
%%%%%%%%%%%%%%%%%%%%%%%%%%%%%%%%%%%%%%%%%%%%%%%%


There currently exists a broad diversity of models that have been proposed for inflation 
\citep{LiddleLyth, Olive, Lyth}. In this section we present the scalar fields as good candidates 
to drive inflation and explain how relate theoretical predictions to observable quantities. 
Here, we limit ourselves to models based on general gravity, i.e. derived from the
Einstein-Hilbert action, and single-field models described by a \textcolor{red}{homogeneous} slow-roll scalar field $\phi$.
\\

Inflation relies on the existence of an early epoch in the universe dominated by a very 
different form of energy; remember the requirement of the unusual property of a negative 
pressure. Such condition can be satisfied by a simple scalar field (spin-0 particle). 
The scalar field which drives the Universe to an inflationary epoch is often termed 
as the \textit{inflaton field}. 
%

Let us consider a scalar field minimally coupled to gravity, with an arbitrary
potential $V(\phi)$ and Lagrangian density $\mathcal{L}$ specified by 


\begin{equation}
S=\int d^4x\, \sqrt{-g}\,\mathcal{L}=\int\, d^4x\, \sqrt{-g}\,
\left[\frac{1}{2}
\partial_{\mu}\phi
\partial^{\mu}\phi -V(\phi)\right].
\end{equation}
%
%
The energy-momentum tensor corresponding to this scalar field Lagrangian is given by
\beq
T_{\mu\nu}=\partial_{\mu}\phi \partial_{\nu}\phi
-g_{\mu\nu}\, \mathcal{L}.
\eeq
%
In the same way as the perfect fluid treatment, the 
energy density $\rho_\phi$ and pressure density $p_\phi$ in FRW metric are found to be 

\begin{eqnarray}
T_{00}=\rho_{\phi}=\frac{1}{2}\dot{\phi}^2 + V(\phi)+ 
\frac{(\nabla \phi)^2}{2a^2},  \\
T_{ii}=p_{\phi}=\frac{1}{2}\dot{\phi}^2 - V(\phi)- \frac{(\nabla
\phi)^2}{6a^2}.
\end{eqnarray}

\noindent
\textcolor{red}{Considering a homogeneous field, its} equation of state corresponding is
 
\begin{equation}
w = \frac{P}{\rho}=\frac{\frac{1}{2}\dot \phi^2-V(\phi)}{\frac{1}{2}\dot \phi^2+V(\phi)}.
\end{equation}

\noindent
We can now split the inflaton field as
\beq \label{eq:split}
\phi({\bf x},t)=\phi_{0}(t)+\delta\phi({\bf x},t),
\eeq
where $\phi_{0}$ is considered a classical field, that is, 
the mean value of the inflaton field on the homogeneous and isotropic state, 
whereas $\delta\phi({\bf x},t)$ describes the quantum fluctuations around $\phi_{0}$.

\noindent
The evolution equation for the background field $\phi_0$  is given by
\begin{equation}
\ddot{\phi_0}+ 3H\dot{\phi_0}= -V'(\phi_0),
\label{eq:motion1}
\end{equation}

\noindent
and moreover, the Friedmann equation (\ref{eq:Friedmann}) with negligible curvature becomes

\beq \label{eq:motion2}
H^2 = \frac{8\pi}{3m^2_{Pl}} \left[{1 \over 2} \dot\phi_0^2 +V(\phi_0)\right],
\eeq
where we have used 
primes as derivatives with respect to the scalar field $\phi_0$. 
\\

 From the structure of the effective energy density and pressure, the acceleration
  equation (\ref{eq:Acce}) becomes, 
 
 \beq
 {\ddot a \over a} = -{8\pi \over m_{Pl}^2}\left(\dot \phi_0^2-V(\phi_0) \right).
 \eeq
 
 \noindent
 Therefore, the inflationary condition to be satisfied is $\dot \phi_0^2 < V(\phi_0)$, and 
 it is easily fulfilled with a suitably flat potential. Now on we will omit the subscript
 `0' by convenience.



\subsection{Slow-roll approximation}
\vskip 6pt
%%%%%%%%%%%%%%%%%%%%%%%%%%%%%%%%%%%%%%%%%%%%%%%%

As we have noted, a period of accelerated expansion can be created by 
the cosmological constant $(\Lambda)$ and hence solve the problems aforementioned.
After a brief period of time, inflation must end up and its energy being converted into conventional
matter/radiation, this process is called \textit{reheating}. In a Universe dominated by a 
cosmological constant the reheating process is seen as $\Lambda$ decaying into 
conventional particles. However, claiming that $\Lambda$ is able to decay is still a 
naive way to face the problem.   
%
On the other hand, scalar fields have the property to behave like a 
\textit{dynamical cosmological constant}. Based on this approach, it is useful to
propose a scalar field model starting with a nearly flat potential, i.e. initially 
satisfies the \textcolor{red}{\textit{first slow-roll}} condition $\dot \phi^2 \ll V(\phi)$. \textcolor{red}{This condition may not necessarily be fulfilled for a long time.
To avoid this problem, the second \textit{slow-roll} condition is defined as $|\ddot{\phi}|\ll |V,_{\phi}|$ or equivalently $|\ddot{\phi}|\ll 3H|\dot{\phi}|$}. In this case the scalar field is slowly rolling 
down its potential, by obvious reasons, such approximation is called \textit{slow-roll} 
\citep{Liddle92, Liddle94}.%  Based on this approach, $\ddot \phi$ is negligible because the Universe is  dominated by the cosmological expansion. 
The equations of motion (\ref{eq:motion1}) 
 and (\ref{eq:motion2}), for slow-roll inflation, then become
 
\bea \label{eq:slow}
3H\dot{\phi} ~~ &\simeq& ~~ -V'(\phi), \\
H^2 ~~ & \simeq& ~~ \frac{8\pi}{3m^2_{Pl}} V(\phi). \label{eq:slow2}
\eea

\noindent
It is easily verifiable that the slow-roll approximation requires the slope 
and curvature of the potential to be small: $V', V'' \ll V$.
\\

The inflationary process can be summarised as an accelerated Universe which takes place when 
the kinetic part of the inflaton field is subdominant over the potential field $V(\phi)$ term. 
Then, when both quantities become comparable the inflationary period ends up given 
rise finally to the reheating process, see Fig. \ref{fig:Field}. 


\begin{figure}[ht] 
\centerline{ \epsfxsize=180pt \epsfbox{Field.pdf} }
\caption{Schematic inflationary process \citep{Baumann}.}
\label{fig:Field}
\end{figure}

\vspace{0.5cm}
It is now useful to introduce the potential slow-roll parameters 
$\epsilon_{\rm v}$ and $\eta_{\rm v}$ in the following way \citep{Liddle92}


\bea
\epsilon_{\rm v}(\phi) &\equiv&{m^2_{Pl} \over 16 \pi } \left({V'\left(\phi\right) \over V\left(\phi\right)}\right)^2, 
\label{eq:epsi} \\
\eta_{\rm v}\left(\phi\right) &\equiv& \frac{m^2_{Pl}}{8\pi} {V''\left(\phi\right) \over V\left(\phi\right)} \label{eq:eta}.
\eea
%
Equations (\ref{eq:slow}) and (\ref{eq:slow2}) are in agreement with the slow-roll approximation
when the following conditions hold

\begin{equation*}
\epsilon_{\rm v}(\phi) \ll 1,  \,\,\,\,\,\,  \mid \eta_{\rm v}(\phi)\mid \ll 1.
\end{equation*}

\noindent
These conditions are sufficient but not necessary, because the validity of the slow-roll
approximations was a requirement in its derivation.
%
The physical meaning of $\epsilon_{\rm v}(\phi)$ can be explicitly seen by expressing equation (\ref{eq:inflation})
 in terms of $\phi$, then, the inflationary condition is equivalent to
 
 \begin{equation}
 {\ddot a \over a} ~~ > ~~ 0 ~~ \Longrightarrow ~~ \epsilon_{\rm v}(\phi) < ~~  1.
\end{equation}

\noindent
Hence, inflation ends up when the value $\epsilon_{\rm v} (\phi_{end})= 1$ is approached.
\\

Within these approximations, it is straightforward to find out the scale factor $a$ between
the beginning and the end of inflation. Because the size of the expansion is 
an enormous quantity, it is useful to compute it in terms of the 
 {\it e}-fold number $N$ defined by 
 
\begin{equation} \label{eq:N}
N \equiv \ln {a(t_{end}) \over a(t)}=
\int_{t}^{t_e}{H\,dt} \simeq 
{8\pi \over m^2_{Pl}} \int_{\phi_e}^{\phi} {V \over V'} d\phi .
\end{equation}

\noindent
To give an estimate of the number of \textit{e}-folds $N$, let us consider the evolution of the Universe 
can be split into different epochs:

\begin{itemize}
\item Inflationary era: horizon crossing ($k=aH$) $\to$ end of inflation $a_{end}$.
\item Radiation era: reheating $\to$ matter-radiation equality $a_{eq}$.
\item Matter era: $a_{eq}$ $\to$ present $a_{0}$.
\end{itemize}

\noindent
Assuming the transition between one era to another is instantaneous, then $N(k)= \ln ({a_k / a_0})$
can be easily computed with:
$$ 
{k\over a_0 H_0}\,=\,{a_k H_k \over a_0 H_0}\,=\,{a_k\over a_{end}}{a_{end}\over a_{reh}}
{a_{reh}\over a_{eq}}{a_{eq} \over a_0}{H_k\over H_0}.
$$
Then, one has \citep{LiddleLyth}

$$
N(k)=62-\ln{k \over a_0 H_0}-\ln{10^{16} GeV \over V_k^{1/4}}+\ln{V_k^{1/4} \over V_{end}}-
{1 \over 3}\ln{V_{end}^{1/4} \over \rho_{reh}^{1/4}}.
$$
%
The last three terms are small quantities related with energy scales during the inflationary 
process and usually can be ignored.
The precise value for the second quantity depends on the model as well as the 
COBE normalisation, however it does not present any significant change to the total 
amount of \textit{e}-folds. Thus, the value for total \textit{e}-foldings is ranged from 50-70 \citep{Lyth}. \textcolor{red}{This value could change if a modification of the full history of the Universe is considered. For instance, thermal inflation can alter $N$ up to a minimum value of N=25 \citep{Lyth1,Lyth2}.}
\\

As we noted, the parameters to describe inflation can be presented
as functions of the scalar field potential. That is, specifying an inflationary 
model with a single scalar field is just selecting an inflationary potential $V(\phi)$. \textcolor{red}{At this point, it is necessary to mention that these inflationary potentials are not chosen arbitrarily. In fact, there is a whole line of research in models of particle physics that are looking for inflationary potentials motivated by fundamental physics. However, for the purposes of this paper, we will not delve into this subject. In this way, every time we mention that we are taking an inflationary potential, it will be understood that this potential is motivated by a fundamental theory}. 
In order to exemplify our point, let us consider the following example.


The potential which describes a massive scalar field is given by:

\beq \label{eq:mass}
V(\phi)= \frac{1}{2}m^2 \phi^2.
\eeq

\noindent
Considering the slow-roll approximation, equations (\ref{eq:motion1}) and (\ref{eq:motion2}) become:
%
\bea
3H\dot \phi &=& -m^2 \phi, \\
H^2 &=& \frac{4\pi m^2 \phi^2}{3 m_{pl}^2}.\nonumber
\eea
\noindent
Thus, the dynamics of this type of model is described by

\bea
\phi(t)&=&\phi_i - \frac{m m_{pl}}{\sqrt{12 \pi}},\\
a(t)&=& a_i \exp\left[\sqrt{\frac{4\pi}{3}}\frac{m}{m_{pl}}\left( \phi_i t - \frac{m m_{pl}}{\sqrt{48 \pi}}t^2 \right) \right], \nonumber
\eea

\noindent
where $\phi_i$ and $a_i$ represent the initial conditions at a given initial time $t=t_i$.
The slow-roll parameters for this particular potential are computed from 
equations (\ref{eq:epsi}) and (\ref{eq:eta})

\beq
\epsilon_{\rm v}=\eta_{\rm v}= \frac{m_{pl}^2}{4\pi} \frac{1}{\phi^2}, 
\eeq

\noindent
that is, an inflationary epoch takes place whilst the condition $|\phi|> {m_{pl}}/\sqrt{4\pi}$ 
is satisfied, and the total amount lapse during this accelerated period is encoded on the e-folds number

\beq
N_{tot}= \frac{2\pi}{m_{pl}^2}\left[\phi^2_i - \phi^2_e \right].
\eeq 

The steps shown before might, in principle, apply to any inflationary single-field model. 
That is, the general information we need to characterised 
cosmological inflation is specified by  only its potential. 

 
%%%%%%%%%%%%%%%%%%%%%%%%%%%%%%%%%%%%%%%%%%%%%%%%
\subsection{Cosmological Perturbations}

Inflationary models have the merit that they do not only explain the 
 homogeneity of the universe on large-scales,
but also provide a theory  for explaining the observed level of {\em
anisotropy}. During the inflationary period, quantum fluctuations of the field 
were driven to scales much larger than the Hubble horizon. Then in this process, the fluctuations
were frozen and turned into metric perturbations \citep{Mukhanov}. 
%
Metric perturbations created during inflation can be described in terms of two types of perturbations.
The {\it scalar, or curvature,} perturbations are coupled with matter in the
universe and form the initial ``seeds'' of structure formation. On the other hand, although the 
{\it tensor perturbations} do not couple to matter, they are associated
to the generation of gravitational waves.
As we shall see, scalar and tensor perturbations are seen as the important components to 
 the CMB anisotropy \citep{Hu}. 
\\

In the same way we have introduced the density parameter for large scales, on small scales
we employ the \textit{density contrast} defined by $\delta \equiv \delta \rho / \rho$.
We now on assume the density contrast for different species in the Universe satisfies the 
\textit{adiabatic conditions}
 \beq
 {1 \over 3}\delta_{{\bf k} b}={1 \over 3}\delta_{{\bf k} c}={1 \over 4}\delta_{{\bf k} \gamma}
 \left(={1 \over 4}\delta_{{\bf k} }\right).
 \eeq 

\noindent
The most general perturbation on the density is described by a linear combination between adiabatic
perturbation as well as \textit{isocurvature perturbation}, which the latter one plays and important role 
when more than one scalar field is considered  \citep{LiddleLyth}.


We introduce the \textit{primordial curvature perturbation} $\mathcal{R}_k(t)$, which has the property 
to be constant within few Hubble times after the horizon exit given by $k=aH$.  
This constant value is called the
\textit{primordial value} and is related with the scalar field perturbation $\delta \phi$ by
\beq
\mathcal{R}_k=-\left[ {H \over \dot \phi}\,\, \delta \phi_k \right]_{k=aH}.
\eeq
%
\noindent
 Then, the  primordial curvature power spectrum $\cal P_{\cal R}(k)$ is computed from
 \beq\label{eq:Pk1}
 \cal P_{\cal R}(k)=\left[\left({H \over \dot \phi} \right)^2 \cal P_{\phi}(k) \right]_{k=aH}.
 \eeq
 
 \noindent
 \textcolor{red}{As it was explained before, if we consider that inflation is at least exponential, then the horizon remains practically constant while all other scales grow. In this way, we can focus on the evolution of the quantum perturbations of the inflaton in a small region compared to the horizon. In this region it is possible to consider the space as locally flat and ignore the metric perturbations. Thus, working in the fourier space, the classical equation of motion of the perturbation part of $\phi({\bf x},t)$ in (\ref{eq:split}) is} 
\beq
(\delta \phi_k)\ddot \,\,+3 H (\delta \phi_k)\dot \,\,+\left({k \over a}\right)^2 \delta \phi_k=0,
\eeq
%
where we have assumed $\delta \phi$ is linear. This basically means that perturbations generated by
vacuum fluctuations have uncorrelated Fourier modes, the signature of \textit{Gaussian perturbations}. 
%

\textcolor{red}{The above equation can be rewritten as the equation of a harmonic oscillator with a variable frequency. If we now move to the quantum world and make the corresponding associations of operators to each classical variable, the quantum dynamics will be determined by (for a detailed explanation see \citep{LiddleLyth2}, page 382)}
\begin{equation}\label{qscalarfield}
\hat{\psi}_k\left(\eta\right)=\frac{\psi_k\left(\eta\right)\hat{a}\left(k\right)+\psi^*_k\left((\eta\right)\hat{a}^\dagger\left(-k\right)}{\left(2\pi\right)^3} \ \ \ \text{with} \ \ \ \psi_k\left(\eta\right)=-\frac{e^{-ik\eta}}{\sqrt{2k}}\frac{k\eta-i}{k\eta},
\end{equation}
\textcolor{red}{where $\hat{a}$ and $\hat{a}^\dagger$ are the particle creation and annihilation operators, $\eta$ is the proper time defined by $\eta\equiv -1/aH$ and $\psi\equiv a\delta\phi$.}

\textcolor{red}{The inflationary process dilutes all possible particles existing before this period. Taking into account this, it is considered that the state in which the system is located is that of the vacuum. Defining the spectrum of perturbations as $\langle\psi_k\psi_k\rangle =2\pi^2 \mathcal{P}_\psi\left(k\right)\delta ^3\left(\vec{k}+\vec{k}'\right)$ and evaluating it a few hubble times after the horizon exit, $\eta=1/aH_k$, with $H_k$ the value of the hubble constant at the time the scale $k$ has left the horizon, it can be shown that} the spectrum is given by
%
\beq\label{eq:Pk2}
\cal P_{\phi}(k)=\left( {H \over 2\pi }\right)^2_{k=aH}.
\eeq
%
Finally, from (\ref{eq:Pk2}) and (\ref{eq:Pk1}) the spectrum of the curvature perturbation is 

\begin{equation}
\cal P_{\cal R}(k) = \left[ \left({H \over \dot \phi}\right) \left({H \over 2 \pi}\right)\right]^2_{k =
a H} .
\end{equation}

\textcolor{red}{On the other hand, we notice that well after horizon exit, $\eta \rightarrow 0$, $\psi_k\left(\eta\right)$ approaches to}
\begin{equation}
\psi_k\left(\eta\right)=-\frac{i}{\sqrt{2k}}\frac{1}{k\eta},
\end{equation}
\textcolor{red}{so that equation \ref{qscalarfield} is rewritten as}
\begin{equation}
\hat{\psi}_k\left(t\right)=\psi_k\left(t\right)\frac{\hat{a}\left(k\right)-\hat{a}^\dagger\left(-k\right)}{\left(2\pi\right)^3}.
\end{equation}
\textcolor{red}{In this way, it can be seen that the temporal dependence of $\hat{\psi}_k$ is now tribial. It implies that once $\psi_k\left(t\right)$ is messured in a time well after horizon exit, it will continue having a definite value. In this way, we can consider that  this quantum fluctuation has become classical once it has crossed the horizon and therefore, the spectrum of quantum perturbations can be taken as the Initial inhomogeneities that will later give rise to the formation of structure. However, as will be seen below, these initial conditions will be slightly modified due to the amount of inflation remaining, once the k-scale has left the horizon.}
 
On the other hand, the creation of gravitational waves corresponds to the tensor 
part of metric perturbation $h_{\mu \nu}$ in (\ref{eq:metric}). In Fourier space, 
tensor perturbation $h_{ij}$ can be expressed as the superposition of two polarisation modes

\beq
h_{ij}= h_{+}\mathit{e}^{+}_{ij}+h_{\times}\mathit{e}^{\times}_{ij},
\eeq
%
where $+$, $\times$ represent the longitudinal and transverse modes.
From Einstein equations it is found that each amplitude $h_{+}$ and $h_{\times}$
behaves as a free scalar field in the sense that
\beq
\psi_{+,\times}\equiv {m_{Pl} \over \sqrt 8}\,\,h_{+,\, \times}.
\eeq 

\noindent
Therefore, \textcolor{red}{taking the results of the scalar perturbations}, each $h_{+,\, \times}$ has a spectrum $\cal P_T$ given by

\beq
\cal P_{T}(k)={8 \over m_{Pl}^2} \left({H \over 2\pi} \right)_{k=aH}^2.
\eeq
%
The canonical normalisation of the field $\psi_{+,\times}$ was chosen such that,
the \textit{ratio of tensor-to-scalar} spectra is  

\begin{equation}
r \equiv {\cal P_{T} \over \cal P_{\cal R}} =16 \epsilon .
\end{equation}

During the horizon exit epoch $k=aH$, $H$ and $\dot \phi$ have
tiny variations during few Hubble times. In this case, the scalar and tensor 
spectra are nearly scale independent and therefore well approximated by power laws 
%
\bea
\cal P_\cal R(k)  &=&\cal P_\cal R(k_0) \left( \frac{k}{k_0} \right)^{n_{\rm s}-1}, \nonumber \\
\cal P_T(k)  &=& \cal P_T(k_0) \left( \frac{k}{k_0} \right)^{n_T} \, .
\eea

\noindent
where the spectral indices are defined as \citep{Lidsey}

\beq
n_{\rm s}-1\equiv {d\ln \cal P_\cal R(k) \over d\ln k},\qquad
n_T \equiv {d\ln \cal P_T(k) \over d\ln k}. 
\eeq

\noindent
A scale-invariant spectrum, called Harrison-Zel'dovich (HZ), has constant variance
on all length scales and it is characterised by $n_{\rm s}= 1$.
 Small deviations from scale-invariance are also considered as a typical signature of the
inflationary models, then 
 the spectral indices $n_{\rm s}$ and $n_{T}$ can be expressed 
 in terms of the slow-roll parameters $\epsilon_{\rm v}$ and $\eta_{\rm v}$, to lowest order, as:

\bea \label{indices}
n_{\rm s} - 1  &\simeq&  - 6~ \epsilon_{\rm v}(\phi) + 2~ \eta_{\rm v}(\phi), \nonumber \\
n_T  &\simeq&  -2~ \epsilon_{\rm v}(\phi). 
\eea

\noindent
The parameters are {\em not} completely independent each other,  but the tensor spectral 
index is proportional to the tensor-to-scalar ratio
$r = -8 n_{T}  $.  This expression is considered as the \textit{consistency
relation} for slow-roll inflation.  
Hence, any inflationary model, to the lowest order in slow-roll, 
can be described in terms of three independent parameters:
 the amplitude of density perturbations
 $\delta_H\equiv 2/5P_\cal R^{1/2} \approx 2 \times 10^{-5}$, the scalar spectral index $n_{\rm s}$,
and the tensor-to-scalar ratio $r$.
In case we need a more accurate description we have to consider higher-order effects, 
and then include parameters for describing the running of scalar ($d n_{\rm s} / d\ln{k}$)
and tensor ($d n_T / d\ln{k}$) index. 
\\

An important point to emphasised is that  $\delta_H$, $r$ and $n_{\rm s}$ are {\em observable} 
parameters that nowadays are tested from several experiments. 
This allows us to compare theoretical predictions with observational data,  
for instance, those provided by the Cosmic Microwave Background radiation. 
In other words, future measurements of these parameters may
probe or at least constrain the inflationary models and therefore the shape of the inflaton potential $V(\phi)$.
\\

Let us back to the massive scalar field example in equation  (\ref{eq:mass}):
%
Inflation ends up when the condition $\epsilon_{\rm v}=1$ is achieved, so $\phi_{end}\simeq m_{pl}/\sqrt{4 \pi}$.
 As we pointed out before, we are interested in models with an $e$-fold number of about $N_{tot}=60$, that is
 \beq
 \phi_i=\phi_{60}\simeq \sqrt{\frac{30}{\pi}}m_{pl}.
 \eeq
 
 \noindent
 Finally, the spectral index and the tensor-scalar ratio for this potential are
 
 \beq
 n_{\rm s}-1= -\frac{m_{pl}^2}{\pi \phi_{60}^2}, \qquad r= \frac{m_{pl}^2}{\phi_{60}^2}.
 \eeq

\noindent
If the massive scalar field potential is the right inflationary model, current observations should favour
the values $n_s\approx 0.97$ and $r\approx 0.1$.
\\



%%%%%%%%%%%%%%%%%%%%%%%%%%%%%%%%%%%%%%%%%%
\section{Inflationary models}
%%%%%%%%%%%%%%%%%%%%%%%%%%%%%%%%%%%%%%%%%%


We have seen that an inflationary model is described by the specification of the 
potential form $V(\phi)$ relevant during inflation. Then, the comparison of inflationary 
model predictions to CMB observations reduces to the following basic steps \citep{Kinney}:
 (1) Given a scalar field potential $V(\phi)$, compute the slow roll parameters $\epsilon_{\rm v}(\phi)$ and
$\eta_{\rm v}(\phi)$. 
(2) Find out $\phi_{end}$ by $\epsilon(\phi_{end})=1$. 
(3) From (\ref{eq:N}), compute the field $\phi_{60}$.
(4)  Compute $r$ and $n_{\rm s}$ as functions of $\phi$, and finally evaluate them at $\phi =
\phi_{60}$ which can be tested by CMB temperature anisotropy data. 
\\

Different types of models are classified by the relation among their slow-roll 
parameters $\epsilon$ and $\eta$, which can be reflected on different relations
 between $r$ and $n_{\rm s}$. Hence, an appropriate parameter space to show the diversity of models 
 is well described by the $n_{\rm s}$---$r$ plane.  


%%%%%%%%%%%%%%%%%%%%%%%%%%%%%%%%%%%%%%%%%

\subsection{Models}


Even if we restrict the analysis to a simple single-field, the number of inflationary models
available is enormous \citep{LiddleLyth, Lyth, Linde05}. Then, it is convenient to classify
different kinds of scalar field potentials following \citep{Kinney2}. 
%

The classification is based on the behaviour of the scalar field potential during inflation.
The three basic types are shown in Figure \ref{fig:models}.
{\em Large field}, the field is initially displaced from an stable minimum and evolves 
towards it. {\em Small field}, the field evolves away from an unstable maximum. 
{\em Hybrid}, the field evolves towards a minimum with vacuum energy different to zero. 


\begin{figure}[ht] 
\centerline{ \epsfxsize=270pt \epsfbox{zoology.pdf} }
\caption{Potential classification. From top to bottom:
\textit{large field, small field and hybrid potential} \citep{Kinney2}.}
\label{fig:models}
\end{figure}
%
A general single field potential can be written in terms of a \textit{height} $\Lambda$ and a 
\textit{width} $\mu$ as
%
\begin{equation}
V\left(\phi\right) = \Lambda^4 f\left({\phi \over \mu}\right).
\end{equation}
%
Different models have different forms for the function $f$.

%%%%%%%%%%%%%%%%%%%%%%%%%%%%%%%%%%%%%%%%
\subsection{Large-field models: $-\epsilon < \eta \leq \epsilon$}
%%%%%%%%%%%%%%%%%%%%%%%%%%%%%%%%%%%%%%%%


\textbf{Large field} models perhaps posses the simplest type of monomial potentials.
These kind of potentials represent the \textit{chaotic} inflationary
scenarios \citep{Linde2}. The distinctive of these models is that the  
shape of the effective potential is not very important in detail. That is, a region
of the Universe where the scalar field is usually situated at $ \phi \sim m_{\rm Pl}$ 
from the minimum of its potential will automatically lead to inflation, see Figure \ref{fig:new1}. Such
models are described by $V''\left(\phi\right) > 0$ and $-\epsilon < \eta
\leq \epsilon$. 
\\

 \begin{figure}
 \begin{center}
  \includegraphics[trim = 20mm 120mm 10mm 40mm, clip, width=7cm, height=4cm]{new2.pdf}
	\caption{Chaotic inflationary potential.}
	\label{fig:new1}
\end{center}	
\end{figure}

\noindent
 A general set of large-field polynomial potentials can be written as
%
\begin{equation}
V\left(\phi\right) = \Lambda^4 \left({\phi \over \mu}\right)^p,
\end{equation}
where it is enough to choose the exponent $p>1$ in order to specify a particular model.
%
This model gives
%\begin{eqnarray}
%n_{\rm s}-1&=&-\frac{4+2p}{4N+p}\, ,\nonumber\\
%r&=&\frac{4\pi p}{4N+p}.
%r&=&\frac{16 p}{4N+p}.
%\end{eqnarray}
\begin{eqnarray}
n_{\rm s}-1&=&-\frac{2+p}{2N}\, ,\nonumber\\
%r&=&\frac{4\pi p}{4N+p}.
r&=&\frac{4 p}{N}.
\end{eqnarray}

\noindent
In this case, gravitational waves can be sufficiently big to eventually be observed $(r\gtrsim 0.1)$.

From the quadratic potential of equation (\ref{eq:mass}), we obtain

%\begin{equation}
%\epsilon \simeq 0.008, \quad \eta \simeq 0.008, \quad n_{\rm s} \simeq 0.97, \quad r \simeq 0.1
%\end{equation}
\begin{equation}
\epsilon \simeq 0.008, \quad \eta \simeq 0.008, \quad n_{\rm s} \simeq 0.97, \quad r \simeq 0.128
\end{equation}

\noindent
In the high power limit the $V \propto \phi^p$ predictions are the same as the exponential 
potential \citep{La}. Hence, a variant of this class of models is 

\begin{equation}
V\left(\phi\right) = \Lambda^4 \exp\left(\phi / \mu\right).
\end{equation}
%
 This type of potential is a rare case presented in inflation, that is because its 
 dynamics has an exact solution given by a power-law expansion.
 For this case the spectral index $n_{\rm s}$ is closely related to the tensor-to-scalar ratio $r$, as 

%\begin{eqnarray}
%n_{\rm s}-1&=&-\frac{m^2_{pl}}{8\pi \mu^2}, \nonumber \\
%r &=& 2\pi \left(1 - n_{\rm s}\right),
%\end{eqnarray}
\begin{eqnarray}
n_{\rm s}-1&=&-\frac{m^2_{pl}}{8\pi \mu^2}, \nonumber \\
r &=& 8 \left(1 - n_{\rm s}\right),
\end{eqnarray}
%
as we observe, the slow-roll parameters are explicitly independent of the \textit{e}-fold number $N$. 

%%%%%%%%%%%%%%%%%%%%
\subsection{Small-field models: $\eta < -\epsilon$}
%%%%%%%%%%%%%%%%%%%%%%%%%%%%%%%%%%


\textbf{Small field} models are typically described by potentials which arise 
naturally from spontaneous symmetry breaking, these type of models are also
known as \textit{new inflation} \citep{Steinhardt, Linde2}. 
In this case, inflation takes place when the field is situated in a false vacuum state,
very close to the top of the hill and rolls down to a stable minimum, see Figure \ref{fig:new2}. 
These models are typically characterized by $V''\left(\phi\right) < 0$
and $\eta < -\epsilon$, usually $\epsilon$ (and hence the tensor amplitude)
is closely zero. 

 \begin{figure}
 \begin{center}
  \includegraphics[trim = 20mm 120mm 10mm 40mm, clip, width=7cm, height=4cm]{new1.pdf}
	\caption{New inflationary potential.}
	\label{fig:new2}
 \end{center}	
\end{figure}

\noindent
Small field potentials, can be written in the generic form as

\begin{equation}
V\left(\phi\right) = \Lambda^4 \left[1 - \left(\phi / \mu\right)^p\right],
\end{equation}
%
where the exponent $p$ differs from model to model. $V(\phi)$ is usualy considered as the 
lowest-order in a Taylor expansion from a more general potential.
In the simplest case of spontaneous symmetry breaking with no special symmetries, 
the dominant term is the mass term, $p = 2$, hence the model gives

%\bea
%n_{\rm s}-1&\simeq& -4\left( \frac{m_{\rm Pl}}{\mu}\right)^2, \nonumber \\
%r &=& 2\pi (1 - n_{\rm s}) \exp\left[- 1 - N\left(1 - n_s\right)\right].
%\eea
\bea
n_{\rm s}-1&\simeq& -4\left( \frac{m_{\rm Pl}}{\mu}\right)^2, \nonumber \\
r &=& 8 (1 - n_{\rm s}) \exp\left[- 1 - N\left(1 - n_s\right)\right].
\eea

%
%
On the other
hand, $p > 2$ has a very different behavior. The scalar spectral index is
\begin{equation}
n_{\rm s}-1 = - {2 \over N} \left({p - 1 \over p - 2}\right),
\end{equation}
independent of $(m_{\rm Pl}/\mu)$. In addition, if it is considered $\mu < m_{\rm Pl}$ 
the values of $r$ are restricted by
%\begin{equation}
%r < 2\pi {p \over N \left(p - 2\right)} \left[{8 \pi \over N p \left(p -
%2\right)}\right]^{p / \left(p - 2\right)}.
%\end{equation}
\begin{equation}
r < 8 {p \over N \left(p - 2\right)} \left[{8 \pi \over N p \left(p -
2\right)}\right]^{p / \left(p - 2\right)}.
\end{equation}



%%%%%%%%%%%%%%%%%%%%%%%%%%%%%%%%%%%%%%%%%%
\subsection{Hybrid models: $0 < \epsilon < \eta$}
%%%%%%%%%%%%%%%%%%%%%%%%%%%%%%%%%%%%%%
%%%%%%%%%%%%%%%%%%%%%%%%%%%%%%%%%%%

The third class called \textbf{hybrid} frequently includes
models which incorporate supersymmetry into inflation \citep{Linde3, Copeland}. 
In these models, the inflaton field $\phi$ evolves
towards a minimum of its potential, however, the minimum has a vacuum energy 
$V(\phi_{\rm min}) = \Lambda^4$ which is different to zero. 
In such cases, inflation continues forever unless an
auxiliary field $\psi$ is added to interact with $\phi$ and ends inflation at some point $\phi = \phi_{\rm c}$. 
 Such models are well described by $V''\left(\phi\right) > 0$ and $0 < \epsilon < \eta$. 
\\

The generic potential for hybrid inflation, in a similar way to large field and small field models
are considered, is
\begin{equation}
V\left(\phi\right) = \Lambda^4 \left[1 + \left(\phi / \mu\right)^p\right].
\end{equation}
%
For $\left({\phi_N / \mu}\right)\gg 1$ the behaviour of
 the large-field models is recovered. Besides that, when $\left({\phi_N /
\mu}\right)\ll 1$, the dynamics is similar to small-field models, but
now the field is evolving towards a dynamical fixed point rather than away from it.  
%
Because the presence of an auxiliary field, the number of {\it e}-folds is

\beq
N(\phi)\simeq \left( {p+1 \over p+2} \right)\left[ {1 \over \eta(\phi_c) }- {1 \over \eta(\phi)}\right]. 
\eeq
%
For $\phi \gg \phi_c$, $N(\phi)$ approaches the value
%
\beq
N_{max}\equiv \left( {p+1 \over p+2} \right) {1 \over \eta (\phi_c)},
\eeq
%
and therefore, the spectral index is given by
 $$
n_{\rm s}-1 \simeq 2 \left( \frac{p+1}{p+2}\right) \frac{1}{N_{max}-N}.
 $$
 %
%
As we can note, the power spectrum is \textit{blue} ($n_{\rm s}>1$) and besides that, the model presents 
a running of the spectral index

\begin{equation}
\label{eq:dsirunning}
{dn_{\rm s} \over d\ln{k}} = -{1 \over 2} \left({p + 2 \over p + 1}\right) 
\left(n_{\rm s} - 1\right)^2.
\end{equation}
%
 This parameter  will be very useful for higher orders and more accurate constraints in 
 future observations. For instance, if it is considered the particular case with $p = 2$ 
 and $n_{\rm s} = 1.2$, the running obtained is $dn_{\rm s} / d\ln{k} = -0.05$ \citep{Kinney3}. 

%%%%%%%%%%%%%%%%%%%%
\subsection{Linear models: $\eta = - \epsilon$}
%%%%%%%%%%%%%%%%%%%%%%%%%%%%%%%%%


Linear models, $V\left(\phi\right) \propto \phi$, are located on the limit between
large field and small field models. They are represented by $V''\left(\phi\right) = 0$ and $\eta =
- \epsilon$. The spectral index and tensor-to-scalar ratio are given by 

\beq
n_{\rm s}-1=-{6\over 4N+1},\qquad
%r = {4\pi \over 4N+1} .
r = {16 \over 4N+1} .
\eeq

\subsection{Other single models}
%%%%%%%%%%%%%%%%%%%%%%%%%%%%%%%%%%%%%%%%%%
%%%%%%%%%%%%%%%%%%%

There still remain several single-field models which cannot fit into this classification, 
for instance the logarithmic potentials \citep{Barrow2}

\beq
V\left(\phi\right) =V_0\left[1+(C g^2/8\pi^2)
\ln\left(\phi/\mu\right)\right].
\eeq
%
Typically they correspond to loop corrections in a supersymmetric theory,
where $C$ denotes the degrees of freedom coupled
to the inflaton and $g$ is a coupling constant.  
%
For this potential, the inflationary parameters are

\bea
n_{\rm s}-1&\simeq& -\frac{1}{N}\left(1+\frac{3C g^2}{16 \pi^2}\right), \nonumber \\
r&\simeq& \frac{1}{N}{Cg^2\over 4\pi}.
 \eea
%
In this model, to end up inflation, an auxiliary field is needed, which is the main feature of
hybrid models. However when it is plotted on the $n_{\rm s}$---$r$ plane, is located into the
small-field region.
\\

\begin{figure}[h!]
\begin{center}
  \includegraphics[trim = 0mm 100mm 0mm 10mm, clip, width=7cm, height=6cm]{Zoo.pdf}
	\caption{Classification of the
potentials in terms of $n_{\rm s}$ and $r$ parameters. }
\label{fig:parameters}
\end{center}
\end{figure}

The classification of inflationary models mentioned previously may be interpreted as an 
arbitrary one. Although, it is very useful because different types of models cover different 
regions of the $(n_{\rm s}, r)$ plane without overlaping, see Figure \ref{fig:parameters}.

\subsection{\textcolor{red}{Multi-field models}}

\textcolor{red}{There are some jobs where it is considered that there is more than one scalar field responsible for generating inflation in the early universe. However, because we are interested just in the last 50-60 e-folds, it is spected that no more than 2 scalar fields can be responsible for this stage. In order to know the way we can work with 2 inflaton scalar fields, we will analize the following example.}

\textcolor{red}{We consider a chaotic-hybrid potential like}
\begin{equation}
V=V_o\left[\left(1-\frac{\psi}{v^2}\right)^2+\frac{\phi^2}{\mu^2}+\frac{\phi^2\psi^2}{\omega^4}\right]
\end{equation}
\textcolor{red}{In the typical hybrid models, it is spected that the \textit{waterfall} scalar field $\psi$ remain in $\psi=0$ while the inflaton field $\phi$ evolve, generating inflation. Then, in a critical value $\phi_c$ the minimum for $\psi=0$ become unstable and the waterfall field roll to its true minimum, finishing immediately with the inflationary era. However, there are some works where it is considered that an amount of non-negligible inflation is generated during this transition, that is usually called \textit{waterfall regimen}. If we consider that at the time when $\phi=\phi_c$ the waterfall field is not zero but $\psi=\psi_o$ due to quantum fluctuations, we will have inflation if it is satisfied}
\begin{subequations}\label{parameterssr}
\begin{equation}\label{parameterssr1}
\epsilon_{\phi_i}=\frac{m_p^2}{16\pi}\left(\frac{V,_{\phi_i}}{V}\right)^2\ll 1, \ \ \ \ 	\eta_{\phi_i \phi_j}=\frac{m_p^2}{8\pi}\frac{V,_{\phi_i\phi_j}}{V}\ll 1, \ \ \ \ i,j=1,2
\end{equation}
\end{subequations}
\textcolor{red}{where $,_{\phi_i}\equiv d/d\phi_i$, and $\phi_1=\phi$ and $\phi_2=\psi$. Then, for this two scalar field model, the inflationary parameters are given by}
\begin{subequations}
\begin{equation}
n_s-1=-6\epsilon_\sigma+2\eta_{\sigma\sigma} \ \ \ \text{and}\ \ \ r=16\epsilon_\sigma,
\end{equation}
\textcolor{red}{where}
\begin{equation}
\epsilon_\sigma=\epsilon_\phi+\epsilon_\psi, \ \ \ \eta_{\sigma\sigma}=\eta_{\phi\phi}\cos^2\theta+2\eta_{\phi\psi}\sin\theta\cos\theta+\eta_{\psi\psi}\sin 2\theta
\end{equation}
\textcolor{red}{and}
\begin{equation}
\cos\theta=\frac{\dot{\phi}}{\sqrt{\dot{\phi}^2+\dot{\psi}^2}}, \ \ \ \sin\theta=\frac{\dot{\psi}}{\sqrt{\dot{\phi}^2+\dot{\psi}^2}}
\end{equation}
\end{subequations}
\textcolor{red}{Despite this model is similar with the hybrid models, when it is plotted on the $n_s-r$ plane, in the same way that the logaritmic potential, it is located into the small-field region if enough amount of inflation is considering in the waterfall regimen.}

%%%%%%%%%%%%%%%%%%%%%%%%%%%%%%%%%%%%%%%%%%%%
\section{Testing inflationary models}
%%%%%%%%%%%%%%%%%%%%%%%%%%%%%%%%%%%%%%%%%%%%


How can observations constrain $n_{\rm s}$ and $r$ in inflationary models?
During several years many projects at different scales of the Universe, have been carried 
out in order to look for observational data to constrain cosmological models. 
Among many others, they are:  
Cosmic Background Explorer (COBE), Wilkinson Microwave Anisotropy Probe (WMAP),
Cosmic Background Imager observations (CBI), Ballon Observations of Millimetric Extra-galactic 
Radiation and Geophysics (BOOMERang), the Luminous Red Galaxy (LRG) subset DR7 of the Sloan
Digital Sky Survey (SDSS), Baryon Acoustic Oscillations (BAO), Supernovae (SNe) data, 
Hubble Space Telescope (HST) and currently the South Pole Telescope (SPT) and the 
Atacama Cosmology Telescope (ACT).
Below, we show some predictions coming from different types of inflationary potentials, 
comparing them with current observational parameters \citep{Mortonson11}.
We mention some results that have been obtained on the 
phase space $n_{\rm s}-r$. At this stage, our interest is mainly focussed 
on the case with no running $dn_{\rm s}/d\ln k =0$.
 \\
 

Figure \ref{fig:Kinney} displays marginalised posterior distributions for $n_{\rm s}$ and $r$ based on
two different types of data sets: WMAP3 by itself, and WMAP3 plus information from the 
LRG subset from SDSS. Considering WMAP3 observations alone \citep{Kinney}, 
the parameters are constrained such that $0.94 < n_{\rm s} < 1.04$
and $r<0.60$ (95\% CL). Those models which present  $n_{\rm s}<0.9$ are  therefore ruled out at
high confidence level. The same is applied for models with $n_{\rm s} > 1.05$. 
%
WMAP data by itself  cannot lead to a strong constraints because the existence of parameter 
degeneracies, like the well known geometrical degeneracy involving $\Omega_m$, 
$\Omega_{\Lambda}$ and $\Omega_k$. However, when it is combined with different types 
of experiments, together they increase the constraining power and might remove degeneracies. 
Furthermore, when the SDSS data are included the limit of the gravitational wave amplitude 
is reduced, whereas the spectral index parameter does not present any relevant change. 
For WMAP3+SDSS the constraints imposed on $n_{\rm s}$ and $r$ are $0.93<n_{\rm s}<1.01$ and
$r<0.31$ \citep{Kinney}. 

\begin{figure}[h!]
\begin{center}
 \includegraphics[trim = 1mm 10mm -10mm 4mm, clip, width=9cm, height=6.5cm]{rn.pdf}
\caption{ WMAP3 data sets constraining 
$n_{\rm s}$ and $r$ parameters. Coloured regions correspond to 68\% and 95\% CL  \citep{Kinney}.}
\label{fig:Kinney}
\end{center}
\end{figure}

On the other hand, Figure \ref{fig:Komatsu} shows that with WMAP5 data alone,   
$r < 0.43$ (95\% CL) while $0.964< n_{\rm s}<1.008$.
When BAO and SN data are added, the limit improves significantly to 
$r < 0.22$ (95\% CL) and $0.953< n_{\rm s}<0.983$.
 \citep{Komatsu}.
\\

\begin{figure}[h!]
\begin{center}
 \includegraphics[trim = 1mm 10mm -10mm -3mm, clip, width=7cm, height=4cm]{komatsu1.pdf}
\caption{Constraints on $n_{\rm s}$ and $r$.
WMAP5 results are coloured blue and WMAP5+BAO+SN red, both on 68\% and 95\% CL \citep{Komatsu}.
}\label{fig:Komatsu}
\end{center}
\end{figure}

Following the same line for inflationary models, we employ the {\sc cosmoMC} package \citep{Lewis}
which allows us to produce some predictions for the $n_{\rm s}$ and $r$ parameters given 
a dataset. To illustrate our point, we consider WMAP seven year data. 
We observe from Figure \ref{fig:infla}, that in order a 
model to be considered as a favourable candidate 
it has to predict a small field with spectral index about  $n_{\rm s}=0.982^{+0.020}_{-0.019}$ 
 and a tensor-to-scalar ratio of $r<0.37$ (95\% CL).
When WMAP-7 is combined with different datasets, the constraints are tighten as is 
shown by \citep{Larson}. 

\begin{figure}[h!]
\begin{center}
 \includegraphics[trim = 10mm 50mm 20mm 40mm, clip, width=8cm, height=6cm]{infla_wmap7.pdf}
\caption{Marginalised probability constraints on $n_{\rm s}$ and $r$ using only WMAP7 data. 
2D constraints are plotted with $1\sigma$ and
$2\sigma$ confidence contours
}\label{fig:infla}
\end{center}
\end{figure}	


 Future observations will reach higher accuracy and therefore strengthen the constraints. 
 We build a simple toy model from an optimistic Planck-like sensitivity \citep{Planck} using 
 the best-fit parameters extracted from WMAP7 year as a fiducial values (see Figure \ref{fig:infla_2}). 
The constraints on $n_{\rm s}$ are highly improved using this idealised experiment: $n_{\rm s}=0.968\pm0.006$
and $r<0.15$ with 95\% CL.

\begin{figure}[h!]
 \begin{center}
 \includegraphics[trim = 1mm 50mm 10mm 40mm, clip, width=6cm, height=4.5cm]{planck.pdf}
\caption{2D marginalised probability onstraints on $n_{\rm s}$ and $r$ for
a particular realisation at Planck-like sensitivity. 2D constraints are plotted with $1\sigma$ adn $2\sigma$
confidence contours.
}\label{fig:infla_2}
\end{center}
\end{figure}	


%%%%%%%%%%%%%%%%%%%%%%%%%%%%%%%%%%%%%%%%%%
\section{CONSTRAINTS ON INFLATIONARY MODELS}
%%%%%%%%%%%%%%%%%%%%%%%%%%%%%%%%%%%%%%%%%%%%%%%%

WMAP3 results are shown in Figure \ref{fig:Kinney2}. Models with $n_{\rm s}=1$ are in a 
good agreement with CMB data. In particular the Harrison-Zel'dovich model: $n_{\rm s}=1, 
r=0, dn_{\rm s}/d \ln k=0$, is not ruled out at more than 95\% CL from CMB data alone.
Similarly, for inflation driven by a massless self-interacting scalar field
$V(\phi) = \lambda\phi^4$,  the contours indicate that this potential
with 60 \textit{e}-folds is still consistent with the WMAP3 data at 95\% CL. 
\\

\begin{figure}[h!] 
\centerline{ \epsfxsize=220pt \epsfbox{zoonorun.pdf} }
\caption{WMAP3 (open contours) and 
WMAP3+SDSS (filled contours) constraints on phase space $n_{\rm s}$, $r$. Contours with 68\% CL 
and 95\% are shown with dashed lines \citep{Kinney}.}
\label{fig:Kinney2}
\end{figure}


On the other hand, WMAP5 results are summarised in Figure \ref{fig:Komatsu2}:
The model $V(\phi)=\lambda \phi^4$, unlike WMAP3 constraints, is found to be located 
far away from the 95\% CL, and therefore it is definitely excluded. For inflation produced by a massive 
scalar field $V(\phi)=(1/2)m^2\phi^2$, the model with $N=50$ is situated outside the 
68\% CL, whereas with $N=60$ is at the boundary of the 68\% CL. 
Therefore, this model for the corresponding number of $e$-folds is consistent with data within the 95\% CL. 
The points represented by $N$-flation describe a model with many massive axion fields \citep{Liddle3}. 
For an exponential potential, it is observed that models with $p<60$ are mainly excluded.
Models with $60<p<70$ are roughly in the boundary of the 95\% region, and $p>70$ are 
in agreement within the 95\% CL. 
Some models with $p\sim 120$ can be located in or outside the 68\% CL, essentially
they lay out in the limit. 
\\

The hybrid potentials, as was already noted, can have different behaviours
depending on the $(\phi/ \mu)$ value. The parameter space can be split into three 
different regions based on $(\phi / \mu)$. For $\phi / \mu \ll 1$ the dynamics is similar to small
fields and the dominant term lays in the region called ``Flat Potential Regime". 
For $\phi / \mu \gg 1$ the result is similar to large field models, this region is called
``Chaotic Inflation-like Regime". The boundary, $\phi / \mu \sim 1$ is named 
``Transition regime". The different $(\phi / \mu)$ values corresponding to their regions
are shown in Figure \ref{fig:Komatsu2}.
  


\begin{figure}[h!]
\centerline{ \epsfxsize=190pt \epsfbox{komatsu2.pdf} }
\caption{Constraints on large and hybrid models obtained from WMAP5+BAO+SN.
They are shown in contours with 68\% and 95\% CL
 \citep{Komatsu}.}
 \label{fig:Komatsu2}
\end{figure}

Two recent experiments have placed new constraints on the cosmological parameters: the Atacama
Cosmology Telescope (ACT; \citet{ACT}) and the South Pole Telescope (SPT; \citet{SPT}).
Figure \ref{fig:SPT} shows the predicted values for a chaotic inflationary model with inflaton
potential $V(\phi)\propto\phi^p$ with 60 e-folds. We observe that models with $p\ge3$
are disfavored at more than 95\% CL.


\begin{figure}[h!]
 \includegraphics[trim = 1mm 1mm 1mm 1mm, clip, width=6cm, height=6cm]{ACT}
  \includegraphics[trim = 1mm 1mm 1mm 1mm, clip, width=6cm, height=6cm]{SPT}
%\centerline{ \epsfxsize=200pt \epsfbox{ACT} }
%\centerline{ \epsfxsize=200pt \epsfbox{SPT} }
\caption{Marginalized 2D probability distribution (68\% and 95\% CL) for the 
tensor-to-scalar ratio $r$, and the scalar spectral index $n_{\rm s}$ for ACT+WMAP (left panel)
and SPT+WMAP (right panel)
 \citep{ACT,SPT}.}
 \label{fig:SPT}
\end{figure}

%%%%%%%%%%%%%%%%%%%%%%%%%%%%%%%%%%%
\section{Conclusions}
%%%%%%%%%%%%%%%%%%%%%%%%%%%%%%%%%%%


\begin{table}[h!]\centering
  \setlength{\tabnotewidth}{1.0\columnwidth}
  \tablecols{3}
   \setlength{\tabcolsep}{2.8\tabcolsep}
\caption{Summarise of the \lowercase{$n_{\rm s}$, $r$} constraints from different measurements \tabnotemark{a}.}
\label{tab:resul}
\begin{tabular}{|c|c|c|}
\toprule
Parameter & Limits & Data set\\
\hline
%& & \\
$n_{\rm s}$& $ 0.968 \pm {0.006}$ & PLANCK (idealised)\\ 
$r$ & $< 0.15$ & \\
\hline
%& & \\
$n_{\rm s}$& $ 0.9711 \pm {0.0099}$ & SPT+WMAP7+BAO+$H_0$\\ 
$r$ & $< 0.17$ & \\
\hline
%& & \\
$n_{\rm s}$& $ 0.970 \pm {0.012}$ & ACT+WMAP7+BAO+$H_0$\\ 
$r$ & $< 0.19$ & \\
\hline
%& & \\
$n_{\rm s}$& $ 0.973 \pm 0.014$ & WMAP7 + BAO +$H_0$\\ 
$r$ & $< 0.24$ & \\
\hline
%& & \\
$n_{\rm s}$& $ 0.982 \pm ^{+0.020}_{-0.019}$ & WMAP7 ONLY\\ 
$r$ & $< 0.36$ & \\
\hline
%& & \\
$n_{\rm s}$& $ 0.968 \pm 0.015$ & WMAP5+BAO+SN\\ 
$r$ & $< 0.22$ & \\
\hline
%& & \\
$n_{\rm s}$ & $0.986\pm  0.022$ & WMAP5 ONLY  \\
$r$ & $  < 0.43 $ &  \\
\hline
%& & \\
$n_{\rm s}$& $0.97\pm 0.04 $ & WMAP3 + SDSS\\ 
$ r$& $<0.31$ & \\
 \hline
%& & \\
$n_{\rm s}$& $0.99 \pm 0.05 $ & WMAP3 ONLY \\
 $r$& $< 0.60 $ &  \\
\bottomrule
\tabnotetext{a}{Peiris et al.2003; Kinney et al.2006; 
Komatsu et al.2009; Komatsu et al.2011; Dunkley et al. 2010; Keisler et al. 2011}
\end{tabular}
%\end{ruledtabular}
\end{table}


Considering the analysis presented here is complicated to prove 
that a given model is correct, since these could be just particular cases of more general 
models with several parameters involved. However, it is possible to eliminate models 
or at least give some constraints on their behaviour leading to a narrower range of study.
\\

Although we have presented some simple examples of potentials, 
the classification in small-field, large-field, and hybrid models is enough to 
cover the entire region of the $n_{\rm s}$--$r$ plane as illustrated in Figure \ref{fig:parameters}.  
Different versions of the three types of models predict qualitatively different
scalar and tensor spectra, so it should be particularly easy to work on
them apart.
\\

We have seen that, the favoured models are those with small $r$ (for $dn_{\rm s}/d\ln{k}\sim 0$)
and slightly \textit{red} spectrum, hence models with \textit{blue} power spectrum  $n_s > 1.001$
are inconsistent with the recently data. This simple but important constraint allows us to rule out
the simplest models corresponding to hybrid inflation of the form 
$V(\phi) = \Lambda^4(1 + (\mu / \phi)^{p})$. There still remain models with red spectra in 
the hybrid classification: inverted models and models with logarithmic potentials. 
\\

Scale-invariant power spectrum $n_{\rm s} = 1$ is consistent within 95\% CL with WMAP3 data 
alone, considering no running of the spectral index. The HZ spectrum 
is therefore not ruled out by WMAP3. However, considering WMAP5 data, 
Figure \ref{fig:Komatsu2} shows that HZ spectrum lays outside 
the 95\% CL region, which indicates it is excluded considering the lowest order 
on the $n_{\rm s}, r$ parameters. When WMAP7 data without tensors is considered, 
scale-invariant spectrum is totally excluded by more than $3 \sigma$, 
 however the inclusion of extra parameters weaken the constraint on the spectral index, in which case 
certain models are still consistent with HZ  even for current observations. 
When chaotic models $V(\phi)\propto\phi^p$ are analysed with current data, 
it is found that quartic models 
($p=4$) are ruled out, whilst models with $p\ge3$ are disfavoured at $>$ 95\% CL.
Moreover, the quadratic potential $V(\phi)= 1/2 m^2 \phi^2$ is in agreement
with all data sets presented here and therefore remains as a good candidate.
Table \ref{tab:resul} summarises the constraints on the $n_{\rm s}$ and $r$ parameters
and its improvements through the years.
Future surveys will  provide a more accurate description of the universe and therefore 
narrow the number of candidates which might better explain the inflationary period.



\section{Acknowledgments }

JAV was supported by CONACyT M\'exico.


%%%%%%%%%%%%%%%%%%%%%%%%%%%%%%%%%%%%%%%%%%%%%%%%
%\bibliography{references}
%%%%%%%%%%%%%%%%%%%%%%%%%%%%%%%%%%%%%%%%%%%%%%%%


\begin{thebibliography}

\bibitem[Albrecht \& Steinhardt(1982)]{Steinhardt} Albrecht, A., and {Steinhardt,} P.~J. 1982, \prl, 48,  1220  

\bibitem[Ambrosio(2002)]{Ambrosio02}Ambrosio, M., et~al. 2002, Eur. Phyis. J. C., 25, 511  

\bibitem[Barrow \& Parsons(1995)]{Barrow2}Barrow, J. D., and {Parsons}, P. 1995, Phys. Rev. D, 52, 10 
  
\bibitem[Barrow \& Tipler(1986)]{Barrow}Barrow, J.~D., Tipler, F.~J., 1986, The Anthropic Cosmological Principle,
  Clarendon Press, Oxford, UK 

 \bibitem[Baumann \& Peiris(2009)]{Baumann}Baumann, D., and  {Peiris,} H. V. 2009, Adv. Sci. Lett., 2, 105
%astro-ph/08103022

\bibitem[Carroll(2001)]{Carrol01}Carroll, S. 2001, Living Rev. Relativity, 3
   
\bibitem[Coles \& Lucchin(1995)]{Coles}Coles, P., and Lucchin, F. 1995, Cosmology, WILEY, England, UK 

 \bibitem[Copeland et al.(1994)]{Copeland} Copeland, E. J., et~al. 1994, Phys. Rev. D, 49, 6410
%astro-ph/93080044

\bibitem[Dodelson (2003)]{Dodelson}Dodelson, S. 2003, Modern Cosmology, Academic Press, Amsterdam, Netherlands 

  \bibitem[Dunkley et al.(2010)]{ACT}Dunkley, J.,  et al. 2010, astro-ph/1009.0866. 
 
\bibitem[Georgi \& Glashow(1974)]{Georgi}Georgi, H., Glashow, S. L. 1974, Phys. Rev. Lett., 32, 438
 
\bibitem[Gold et al.(2011)]{Gold}Gold, B., et al. 2011, ApJS, 192, 15

\bibitem[Guth(1997)]{Guth2}Guth, A.~H. 1997, The Inflationary Universe,  Ed. Vintage
        
\bibitem[Guth(1981)]{Guth}Guth, A.~H. 1981, \prd, 23,  347 

\bibitem[Hu \& Dodelson(2002)]{Hu}Hu, W., and {Dodelson,} S. 2002, Annu. Rev. Astron. and Astrophys., 40, 171 
 
\bibitem[Hinshaw et al.(2009)]{wmap5}Hinshaw, G., et~al. 2009, ApJS, 180, 225
%astro-ph/08030732 

  \bibitem[Keisler et al.(2011)]{SPT}Keisler, R., et al. 2011, astro-ph/1105.3182  
 
\bibitem[Kinney(2004)]{Kinney2}  Kinney, W . H. 2004, CU-TP-1083, astro-ph/0301448
 
\bibitem[Kinney et al.(2006)]{Kinney} Kinney, W. H.,  et~al. 2006, Phys. Rev. D, 74, 023502
% astro-ph/0605338 

\bibitem[Kinney \& Riotto(1998)]{Kinney3} Kinney, W.~H., and {Riotto}, A.,  1998, Phys. Lett., 435B, 272
%astro-ph/9802443

\bibitem[Kolb \& Turner(1983)]{Kolb83}Kolb, E. W., Turner, M. S. 1983, Ann Rev Nucl Part Sci. 33, 645
  
\bibitem[Kolb \& Turner(1994)]{Kolbbo}Kolb, E. W., Turner, M.S. 1994, The Early Universe, Westview Press  

 \bibitem[Komatsu(2009)]{Komatsu} Komatsu, E., et~al. 2009, ApJS., 180, 330 
%astro-ph/08030547
  
\bibitem[Komatsu et al.(2011)]{Komat}Komatsu, E., et~al. 2011, Astrophys. J. Suppl., 192, 18

  \bibitem[La \& Steinhardt(1999)]{La}La, D., and {Steinhardt}, P.J.  1999, Phys. Rev. D, 59, 064029
 
\bibitem[Larson et al.(2011)]{Larson}Larson, D., et~al. 2011, ApJS., 192, 16

\bibitem[Lewis \& Bridle(2002)]{Lewis}Lewis A., and Bridle, S., 2002, Phys. Rev D, 66, 103511 
   
 \bibitem[Liddle(1999)]{Liddle2} Liddle, A. 1999, AIP Conf. Proc., 476, 11
% astro-ph/9901041

 \bibitem[Liddle(1998)]{Liddle3}Liddle, A.,  {Mazundar,} A., and {Schunck,} F. E. 1998, Phys. Rev. D, 58, 061301  
%astro-ph/08030547

\bibitem[Liddle(1999)]{Liddle}Liddle, A. 1999, An introduction to Modern Cosmology, WILEY, England, UK

\bibitem[Liddle \& Lyth(2000)]{LiddleLyth} Liddle, A.~R., and  Lyth, D.~H. 2000, Cosmological inflation and large-scale structure,
  Cambridge University Press, Cambridge, UK 

\bibitem[Liddle \& Lyth(2009)]{LiddleLyth2} Liddle, A.~R., and  Lyth, D.~H. 2009, The primordial density perturbation,
  Cambridge University Press, Cambridge, UK 
  
 \bibitem[Liddle \& Lyth(1992)]{Liddle92}Liddle, A.R. and Lyth, D.H 1992, Phys. Lett. B, 291, 39

 \bibitem[Liddle et al.(1994)]{Liddle94} Liddle, A. R., et~al. 1994, Phys. Rev. D, 50, 12
%astro-ph/0302225
  
\bibitem[Linde(1982)]{Linde}Linde, A.~D. 1982, Phys. Lett. B, 108,  389 
 
 \bibitem[Linde(1983)]{Linde2}Linde, A.~D. 1983, Phys. Lett. B, 129, 177 

\bibitem[Linde(1990)]{Lindeb} Linde, A.~D. 1990, Particle Physics and Inflationary Cosmology, Harwood Academic,
Switzerland 

 \bibitem[Linde(1991)]{Linde3}Linde, A. D. 1991, Phys. Lett. B,  259, 38
 
\bibitem[Linde(2005)]{Linde05}{Linde,} A. 2005, J.Phys.Conf.Ser., 24 
 
\bibitem[Lidsey(1997)]{Lidsey}  Lidsey, J. E., et al. 1997, Annu. Rev.Mod.Phys., 69,  373  
 
\bibitem[Lyth \& Riotto(1999)]{Lyth}Lyth, D.~H., and {Riotto,} A. 1999, Physics Reports, 314, 1

\bibitem[Lyth \& Stewart(1995)]{Lyth1}Lyth, D.~H., and {Stewat,} E.~D., 1995, Physics Rev. Lett. 75, 201, hep-ph/9502417

\bibitem[Lyth \& Stewart(1996a)]{Lyth2}Lyth, D.~H., and {Stewat,} E.~D., 1996a, Physics Rev. D 53, 1784, hep-ph/9510204

\bibitem[Mortonson et al.(2011)]{Mortonson11}Mortonson, M. J., Peiris, H.V., and Easther, R. 2011, Phys.Rev. D, 83, 043505 
  
 \bibitem[Mukhanov \& Chibisov(1997)]{Mukhanov}Mukhanov, V. F., and {Chibisov,} G. V. 1997, JETP Letters, 33, 532
  
\bibitem[Olive(1990)]{Olive} Olive, K. ~A. 1990, Physics Reports, 190, 307  

\bibitem[Peiris et al.(2003)]{Peiris}Peiris, H. V., et~al. 2003, ApJS., 148, 213  
%astro-ph/0302225
 
   \bibitem[Planck Collaboration(2006)]{Planck}The Planck Collaboration, 2006, astro-ph/0604069. 
 
\bibitem[Riess(2009)]{Riess} Riess, A.~G., et~al. 2009, ApJ., 699, 539 
% astro-ph/0210007 

\bibitem[Smooth et al.(1992)]{Smooth}Smooth, G. F., et~al. 1992, ApJ Letters, 396, L1 
  
\bibitem[Springel et al.(2005)]{Sping} Springel, V., et~al.  2005, Nature, 435, 629  

\bibitem[Tegmark  et al.(2001)]{Teg}Tegmark, M., et~al. 2001, Phys. Rev. D, 63, 043007   

 \bibitem[Vilenkin \& Shellard(2000)]{Vilenkin}Vilenkin, A., Shellard, E. P. S. 2000, Cosmic Strings and Other Topological Defects,
Cambridge University Press,  Cambridge, UK 



%astro-ph/1001.4538

\end{thebibliography}


\end{document}
